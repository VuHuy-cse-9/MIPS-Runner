\documentclass[10pt]{article}
\usepackage[top = 1in, bottom = 1in, left = 1in, right = 1in]{geometry}
\usepackage{graphicx}
\usepackage{fancyhdr}


\pagestyle{fancy}
\rhead{TickLab}
\lhead{MIPS Runner}


\begin{document}
\begin{Large}
\begin{center}
\textbf{TICKLAB}
\end{center}
\begin{center}
\textbf{Documentary}
\includegraphics[scale = 0.4]{tickLabLogo1.png}
\end{center}
\end{Large}

\begin{flushleft}
\begin{itemize}
\begin{Large}
\rule[10pt]{\textwidth}{1pt}
\begin{Huge}
\textbf{MIPS RUNNER}
\end{Huge}
\rule{\textwidth}{1pt}

\vspace{1cm}
\item[] \textbf{Developer: }
	\begin{itemize}
	\item[] Nguyen Thanh Toan
	\item[] Vu Nguyen Minh Huy 
	\end{itemize}
\item[]\textbf{Instructor:}
\begin{enumerate}
\item[]	Huynh Hoang Kha
\end{enumerate}

\end{Large}
\end{itemize}
\end{flushleft}

\begin{center}
\begin{Large}December 15th 2019 \end{Large}
\end{center}
\newpage

\begin{Large}
\begin{flushleft}
\textbf{Content}
\end{flushleft}
\begin{enumerate}
\item \textbf{Idea:.....................................................}3
	\begin{itemize}
	\item Initial idea:......................................................3
	\item Main Idea:.......................................................3
	\end{itemize}
\item \textbf{Researching and doing process:............4}
	\begin{itemize}
	 \item Knowledge for research:...................................4
	\item Figure:.............................................................4
	\item Processing:......................................................5
	\end{itemize}
\item \textbf{Product:..............................................7}
	\begin{itemize}
	\item Principle of operation:.....................................7
	\item demonstration:................................................9
	\end{itemize}
\item \textbf{Conclusion:.........................................}12
	\begin{itemize}
	\item Summary:......................................................12
	\item Evaluating working processing:......................12
	\item Lesson after project........................................12
	\end{itemize}
\end{enumerate}
\end{Large}

\newpage


\begin{enumerate}
\begin{Large} \item \textbf{Idea} \end{Large}
	\begin{enumerate}
	\begin{large}
	\item[1.1] \textbf{Initital idea:}
	\item[-]Assembly is a low-level programming language that affects registers directly. The execution of an assembly program will affect the operating status of the computer. This leads to the need for a virtual machine to simulate assembly program behavior. The popular virtual machine is the Java virtual machine.
	\item[-]MIPS-Setting low-level machine language on c ++ that can translate and show users running each command
	\item[1.2] \textbf{Main idea:}
	\item[-]Write a compile program to check for code syntax errors
	\item[-]Identify and save labels to perform address jump commands
	\item[-]Simulate components with 32 bit memory cells, store values
	\item[-]After compile, put the register data register on the screen and each command (10 sentences) for the user to manipulate to run each command line
	\end{large}
	\end{enumerate}
\end{enumerate}

\begin{enumerate}
\begin{Large}
\item[2.]\textbf{Researching and doing process}
\end{Large}
	\begin{enumerate}
	\begin{large}
	\item[2.1]\textbf{Knowledge for research:}
	\item[a)]\textit{Document}:
	\begin{enumerate}
	\item[-]Computer organization and design 5th edition
	\item[-]Mr Dat's report
	\item[-]Web documents
	\begin{enumerate}
	\begin{normalsize}
	\item[+]\texttt{http://www.mrc.uidaho.edu/mrc/people/jff/digital/MIPSir.html}
	\item[+]\texttt{http://www.cs.uwm.edu/classes/cs315/Bacon/Lecture/HTML/ch05s04.html}
	\item[+]\texttt{https://people.cs.pitt.edu/~childers/CS0447/lectures/SlidesLab92Up.pdf}
	\end{normalsize}
	\end{enumerate}
	\end{enumerate}
	\item[b)]\textit{Instructor:}
	\item[]Huynh Hoang Kha
	\item[c)]\textit{The process of inquiry:}
	\begin{enumerate}
	\item[-]Some noticeable things:
	\begin{enumerate}
	\item[+]Register
	\item[+]Instruction
	\item[+]The way computer executes variables
	\item[]Initialize variable - load value into memory -Store in register - executes.
	\item[+]How the computer processes loops by saving the address of the statement with a label
	\end{enumerate}
	\end{enumerate}
	\item[2.2.]\textbf{Figure:}
\begin{flushleft}
\includegraphics[scale=0.3]{sodo.png}
\end{flushleft}
	\item[2.3.]\textbf{Processing:}
	\begin{enumerate}
	\item[a)]\textit{First outline:}
	\item[-]Read the code file from a text file
	\item[-]Split into lines in the textProcessor file
	\item[-]Split each line into tokens in the tokenList file
	\item[-]Identify the tokens (Instruction, label) in the InstructionOperand file separately the data envelope (.data) and the commandtext area: will be processed separately to store variable labels
	\item[-]Instruction is divided into 4 types
	\begin{enumerate}
	\item[+]ThreeArgInstruction(three argument instruction)
	\item[+]TwoArgInstruction(two argument instruction)
	\item[+]OneArgInstruction(one argument instruction)
	\item[+]ZeroArgInstruction(zero argument instruction)
	\end{enumerate}
	\item[-]Each type of instruction has corresponding instructions which execute those instructions.
	\item[-]Register:
	\begin{enumerate}
	\item[+]Registers are stored in the memoryManager file
	\item[+]Simulate registers into two array functions of 35 elements and 31 elements (floating-point register) to store values
	\item[+]Particularly register value pc, is the order of lines commands are processed in the textProcessor file
	\end{enumerate}
	\item[*]Difficulty:
	\begin{enumerate}
	\item[-]To recognize token, when handling each line, tokens that do not recognize it is a immediate value or register values
	\item[-]The idea is only suitable when the labels are declared before the jump or jal command to the labels
	\item[-]Before processing the line must have a compile
	\end{enumerate}
	\item[b)]\textit{Final idea:}
	\item[-]Right from the textProcessor file, identify the tokens as Immediate value, register, label, variable label
	\item[-]Run the program before once to save the labels
	\item[-]Add compile feature
	\end{enumerate}
	\end{large}
	\end{enumerate}
\end{enumerate}

\begin{enumerate}
\begin{Large}
\item[3.] \textbf{Product}
\end{Large}
\begin{large}
\begin{enumerate}
\item[3.1.]\textbf{Priciple of operation:}
\item[-]Read a text file (source.asm) from textProcessor class
Then separate these sourecode line by line, save these lines to src (char **) to save each char * corresponding to 1 statement
\item[-]Parse sourceCode into Instruction:
\begin{enumerate}
\item[+]Partition to declare data (.data) and write code (.text)
\item[+]Identify the label
\item[+]Identify Instruction:
\end{enumerate}
\begin{enumerate}
\item[.]Use the register pc to save the order of the current statement in the char** array
\item[.]Use tokenList to split elements in the line into tokens, saved in token* array 
\item[.]Counts the number of elements in tokenList * to identify which Instruction type (Three, Two, One or Zero instruction) is
\item[.]Compile:If the number of arguments in tokenList is more than 3, compile error
\end{enumerate}
\item[-]Processing Instruction: e.g ThreeArgInstruction:
\begin{enumerate}
\item[+]Simulate 3 variables rd, rs, rt in class InstructionOperand.
\item[+]This Instruction Operand class is responsible for assigning identification 7 symbols:
\begin{enumerate}
\item[.]Variable label
\item[.]Label
\item[.]Register
\item[.]Floating-point register
\item[.]Integer
\item[.]Float
\item[.]Address
\end{enumerate}
\item[+]Identify the Instruction in ThreeArgInstruction:
\begin{enumerate}
\item[.]Use the variables rs, rt, rd to access memory if it is a register (include memoryManager into InstructionOperand, use the pointer to access array to store Register data)
\item[.]Compile: If in that Instruction, the identifiers of rs, rt cannot be exceeded (e.g. addi requires rt as immediate value = integer), an error will be displayed
\item[.]If it is immediate value, simulate a new variable different from the register memory cell
\end{enumerate}
\item[+]Identify the Instruction in OneArgInstruction:
\begin{enumerate}
\item[.]The labels have been saved the original address, when jumping, just change pc (saving the current location)
\end{enumerate}
\end{enumerate}
\item[-]Export values to the screen using UImanager class.

\item[3.2]\textbf{Demonstration:}
\end{enumerate}
\end{large}
\end{enumerate}
\begin{center}
\includegraphics[scale=0.35]{assembly.png}
Assembly
\includegraphics[scale=0.35]{sourceCodeImage.png}
put Assembly on display
\includegraphics[scale=0.15]{compile.jpg}
compile error
\includegraphics[scale=0.35]{runline.png}
\begin{enumerate}
press 'n' to execute next line
\item[]press 'u' to show line are executing (not executing)
\end{enumerate}

\includegraphics[scale=0.35]{register.png}
press 'r' to show register memory
\includegraphics[scale=0.35]{memory.png}
press 'v' to show variable memory
\end{center}

\begin{enumerate}
\begin{Large}
\item[4.]\textbf{Conclusion}
\begin{enumerate}
\begin{large}
\item[4.1.]\textbf{Summary:}
\item[-] Completed most of comments of assembly
\item[-]Comprehended how to write an assembly
\item[-]Understand how a computer operated, oraganised and designed
\item[4.2.]\textbf{Evaluating working processing:}
\item[-]Applying Git for operating project
\item[-]To aware the function of pare coding
\item[-]Frequent Interaction for planning idea,evaluating progress
\item[-]Comprehended the worthy value of using class,enhanced ability of applying object oriented programming
\item[4.3.]\textbf{Lesson after project:}
\item[-]Drawing an sufficient picture clearly before starting to code
\item[-]Practicing enhancing ability of algorithm for dealing with problems fast,adaptive for fixing after.
\end{large}
\end{enumerate}
\end{Large}
\begin{center}
This is the end of the report.
\end{center}
\end{enumerate}

\end{document}